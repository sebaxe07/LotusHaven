% A LaTeX template for MSc Thesis submissions to 
% Politecnico di Milano (PoliMi) - School of Industrial and Information Engineering
%
% S. Bonetti, A. Gruttadauria, G. Mescolini, A. Zingaro
% e-mail: template-tesi-ingind@polimi.it
%
% Last Revision: October 2021
%
% Copyright 2021 Politecnico di Milano, Italy. NC-BY

\documentclass{Configuration_Files/PoliMi3i}

%------------------------------------------------------------------------------
%	REQUIRED PACKAGES AND  CONFIGURATIONS
%------------------------------------------------------------------------------

% CONFIGURATIONS
\usepackage{parskip} % For paragraph layout
\usepackage{setspace} % For using single or double spacing
\usepackage{emptypage} % To insert empty pages
\usepackage{multicol} % To write in multiple columns (executive summary)
\setlength\columnsep{15pt} % Column separation in executive summary
\setlength\parindent{0pt} % Indentation
\raggedbottom  

% Reduce underfull \hbox warnings
\hbadness=10000 % Silences the underfull \hbox warnings
\hfuzz=15pt     % Silences overfull \hbox warnings below 15pt
\vfuzz=15pt     % Silences overfull \vbox warnings below 15pt
\tolerance=1000 % Makes TeX try harder to fit text

% PACKAGES FOR TITLES
\usepackage{titlesec}
% \titlespacing{\section}{left spacing}{before spacing}{after spacing}
\titlespacing{\section}{0pt}{3.3ex}{2ex}
\titlespacing{\subsection}{0pt}{3.3ex}{1.65ex}
\titlespacing{\subsubsection}{0pt}{3.3ex}{1ex}
\usepackage{color}

% PACKAGES FOR LANGUAGE AND FONT
\usepackage[english]{babel} % The document is in English  
\usepackage[utf8]{inputenc} % UTF8 encoding
\usepackage[T1]{fontenc} % Font encoding
\usepackage[11pt]{moresize} % Big fonts

% PACKAGES FOR IMAGES
\usepackage{graphicx}
\usepackage{transparent} % Enables transparent images
\usepackage{eso-pic} % For the background picture on the title page
\usepackage{subcaption} % Numbered and captioned subfigures using subcaption package.
\usepackage{tikz} % A package for high-quality hand-made figures.
\usetikzlibrary{}
\graphicspath{{./Images/}} % Directory of the images
\usepackage{caption} % Coloured captions
\usepackage{xcolor} % Coloured captions
\usepackage{amsthm,thmtools,xcolor} % Coloured "Theorem"
\usepackage{float}

% STANDARD MATH PACKAGES
\usepackage{amsmath}
\usepackage{amsthm}
\usepackage{amssymb}
\usepackage{amsfonts}
\usepackage{bm}
\usepackage[overload]{empheq} % For braced-style systems of equations.
\usepackage{fix-cm} % To override original LaTeX restrictions on sizes

% PACKAGES FOR TABLES
\usepackage{tabularx}
\usepackage{longtable} % Tables that can span several pages
\usepackage{colortbl}

% PACKAGES FOR ALGORITHMS (PSEUDO-CODE)
\usepackage{algorithm}
\usepackage{algorithmic}

% PACKAGES FOR REFERENCES & BIBLIOGRAPHY
\usepackage[colorlinks=true,linkcolor=black,anchorcolor=black,citecolor=black,filecolor=black,menucolor=black,runcolor=black,urlcolor=black]{hyperref} % Adds clickable links at references
\usepackage{cleveref}
\usepackage[square, numbers, sort&compress]{natbib} % Square brackets, citing references with numbers, citations sorted by appearance in the text and compressed
\bibliographystyle{abbrvnat} % You may use a different style adapted to your field

% OTHER PACKAGES
\usepackage{pdfpages} % To include a pdf file
\usepackage{afterpage}
\usepackage{lipsum} % DUMMY PACKAGE
\usepackage{fancyhdr} % For the headers
\fancyhf{}

% Input of configuration file. Do not change config.tex file unless you really know what you are doing. 
% Define blue color typical of polimi
\definecolor{bluepoli}{cmyk}{0.4,0.1,0,0.4}

% Custom theorem environments
\declaretheoremstyle[
  headfont=\color{bluepoli}\normalfont\bfseries,
  bodyfont=\color{black}\normalfont\itshape,
]{colored}

% Set-up caption colors
\captionsetup[figure]{labelfont={color=bluepoli}} % Set colour of the captions
\captionsetup[table]{labelfont={color=bluepoli}} % Set colour of the captions
\captionsetup[algorithm]{labelfont={color=bluepoli}} % Set colour of the captions

\theoremstyle{colored}
\newtheorem{theorem}{Theorem}[chapter]
\newtheorem{proposition}{Proposition}[chapter]

% Enhances the features of the standard "table" and "tabular" environments.
\newcommand\T{\rule{0pt}{2.6ex}}
\newcommand\B{\rule[-1.2ex]{0pt}{0pt}}

% Pseudo-code algorithm descriptions.
\newcounter{algsubstate}
\renewcommand{\thealgsubstate}{\alph{algsubstate}}
\newenvironment{algsubstates}
  {\setcounter{algsubstate}{0}%
   \renewcommand{\STATE}{%
     \stepcounter{algsubstate}%
     \Statex {\small\thealgsubstate:}\space}}
  {}

% New font size
\newcommand\numfontsize{\@setfontsize\Huge{200}{60}}

% Title format: chapter
\titleformat{\chapter}[hang]{
\fontsize{50}{20}\selectfont\bfseries\filright}{\textcolor{bluepoli} \thechapter\hsp\hspace{2mm}\textcolor{bluepoli}{|   }\hsp}{0pt}{\huge\bfseries \textcolor{bluepoli}
}

% Title format: section
\titleformat{\section}
{\color{bluepoli}\normalfont\Large\bfseries}
{\color{bluepoli}\thesection.}{1em}{}

% Title format: subsection
\titleformat{\subsection}
{\color{bluepoli}\normalfont\large\bfseries}
{\color{bluepoli}\thesubsection.}{1em}{}

% Title format: subsubsection
\titleformat{\subsubsection}
{\color{bluepoli}\normalfont\large\bfseries}
{\color{bluepoli}\thesubsubsection.}{1em}{}

% Shortening for setting no horizontal-spacing
\newcommand{\hsp}{\hspace{0pt}}

\makeatletter
% Renewcommand: cleardoublepage including the background pic
\renewcommand*\cleardoublepage{%
  \clearpage\if@twoside\ifodd\c@page\else
  \null
  \AddToShipoutPicture*{\BackgroundPic}
  \thispagestyle{empty}%
  \newpage
  \if@twocolumn\hbox{}\newpage\fi\fi\fi}
\makeatother

%For correctly numbering algorithms
\numberwithin{algorithm}{chapter}

%----------------------------------------------------------------------------
%	NEW COMMANDS DEFINED
%----------------------------------------------------------------------------

% EXAMPLES OF NEW COMMANDS
\newcommand{\bea}{\begin{eqnarray}} % Shortcut for equation arrays
\newcommand{\eea}{\end{eqnarray}}
\newcommand{\e}[1]{\times 10^{#1}}  % Powers of 10 notation

%----------------------------------------------------------------------------
%	ADD YOUR PACKAGES (be careful of package interaction)
%----------------------------------------------------------------------------

%----------------------------------------------------------------------------
%	ADD YOUR DEFINITIONS AND COMMANDS (be careful of existing commands)
%----------------------------------------------------------------------------

%----------------------------------------------------------------------------
%	BEGIN OF YOUR DOCUMENT
%----------------------------------------------------------------------------

\begin{document}

\fancypagestyle{plain}{%
\fancyhf{} % Clear all header and footer fields
\fancyhead[RO,RE]{\thepage} %RO=right odd, RE=right even
\renewcommand{\headrulewidth}{0pt}
\renewcommand{\footrulewidth}{0pt}}

%----------------------------------------------------------------------------
%	TITLE PAGE
%----------------------------------------------------------------------------

\pagestyle{empty} % No page numbers
\frontmatter % Use roman page numbering style (i, ii, iii, iv...) for the preamble pages

\puttitle{
	title=Title, % Title of the thesis
	name=Name Surname, % Author Name and Surname
	course=Xxxxxxx Engineering - Ingegneria Xxxxxxx, % Study Programme (in Italian)
	ID  = 000000,  % Student ID number (numero di matricola)
	advisor= Prof. Name Surname, % Supervisor name
	coadvisor={Name Surname, Name Surname}, % Co-Supervisor name, remove this line if there is none
	academicyear={20XX-XX},  % Academic Year
} % These info will be put into your Title page 

%----------------------------------------------------------------------------
%	PREAMBLE PAGES: ABSTRACT (inglese e italiano), EXECUTIVE SUMMARY
%----------------------------------------------------------------------------
\startpreamble
\setcounter{page}{1} % Set page counter to 1

% ABSTRACT IN ENGLISH
\chapter*{Abstract} 
Here goes the Abstract in English of your thesis followed by a list of keywords.
The Abstract is a concise summary of the content of the thesis (single page of text)
and a guide to the most important contributions included in your thesis.
The Abstract is the very last thing you write.
It should be a self-contained text and should be clear to someone who hasn't (yet) read the whole manuscript.
The Abstract should contain the answers to the main scientific questions that have been addressed in your thesis.
It needs to summarize the adopted motivations and the adopted methodological approach as well as the findings of your work and their relevance and impact.
The Abstract is the part appearing in the record of your thesis inside POLITesi,
the Digital Archive of PhD and Master Theses (Laurea Magistrale) of Politecnico di Milano.
The Abstract will be followed by a list of four to six keywords.
Keywords are a tool to help indexers and search engines to find relevant documents.
To be relevant and effective, keywords must be chosen carefully.
They should represent the content of your work and be specific to your field or sub-field.
Keywords may be a single word or two to four words. 
\\
\\
\textbf{Keywords:} here, the keywords, of your thesis % Keywords

%----------------------------------------------------------------------------
%	LIST OF CONTENTS/FIGURES/TABLES/SYMBOLS
%----------------------------------------------------------------------------

% TABLE OF CONTENTS
\thispagestyle{empty}
\tableofcontents % Table of contents 
\thispagestyle{empty}
\cleardoublepage

%-------------------------------------------------------------------------
%	THESIS MAIN TEXT
%-------------------------------------------------------------------------
% In the main text of your thesis you can write the chapters in two different ways:
%
%(1) As presented in this template you can write:
%    \chapter{Title of the chapter}
%    *body of the chapter*
%
%(2) You can write your chapter in a separated .tex file and then include it in the main file with the following command:
%    \chapter{Title of the chapter}
%    \input{chapter_file.tex}
%
% Especially for long thesis, we recommend you the second option.

\addtocontents{toc}{\vspace{2em}} % Add a gap in the Contents, for aesthetics
\mainmatter % Begin numeric (1,2,3...) page numbering

% Main chapters
\chapter{Introduction}
% chktex-file 44
% --------------------------------------------------
% 1 Introduction
% --------------------------------------------------

\section{Group Information}

\textbf{Group Name:}~HyperDaGiaZeySe

\textbf{Group Composition:}

\begin{table}[H]
    \centering
    \caption*{\textbf{Group Composition}}
    \begin{tabular}{|l|l|}
        \hline
        \rowcolor{bluepoli!20}
        \textbf{Name Surname} & \textbf{Person Code} \\
        \hline
        Zeynep Erbaysal & 11035715 \\
        Sebastian Perea & 10986638 \\
        Daniel Ruiz & 10988760 \\
        Giacomo Scampini & 10764570 \\
        \hline
    \end{tabular}
\end{table}

\section{Project Information}

\textbf{Website:}~\url{https://lotus-haven.vercel.app/}~\\
\textbf{GitHub Repository:}~\url{https://github.com/sebaxe07/LotusHaven}

\section{Work Breakdown}

Throughout the project, we aimed to maintain a balanced and collaborative workflow, ensuring that each team member could contribute and learn from different aspects of the development process. As the project progressed, we naturally divided responsibilities to leverage individual strengths and improve efficiency. Sebastian Perea and Daniel Ruiz primarily focused on the core development and implementation of the application, while Zeynep Erbaysal and Giacomo Scampini concentrated on the website’s styling and user interface design. Despite this division, all members participated in discussions and contributed ideas to both the technical and design components, resulting in a cohesive and well-rounded final product.

\chapter{Documentation}
% --------------------------------------------------
% 2 Documentation
% --------------------------------------------------

\section{Chosen Theme}
This project presents the design and development of a yoga center website focused on promoting wellness, clarity, and connection. The website architecture is built around key entities such as instructors, activity types, and scheduled sessions, with main navigational pages including the homepage, About Us, and Contact Us. The design process included structured content organization, detailed wireframes with design rationale, and realistic user interaction scenarios to ensure intuitive navigation. The underlying database schema supports efficient data flow and user interaction, providing a cohesive and user-friendly experience for visitors seeking information about yoga activities and instructors.

\section{Technological Choices}

\begin{itemize}
    \item \textbf{Hosting: Vercel} -- The website is deployed and hosted on Vercel, a cloud platform optimized for frontend frameworks and static sites. Vercel provides seamless integration with GitHub for continuous deployment, fast global CDN, and automatic SSL, ensuring high availability and performance.
    \item \textbf{Database: Supabase} -- Supabase is used as the backend database solution. It offers a scalable, open-source alternative to Firebase, providing a PostgreSQL database, authentication, and real-time capabilities. Supabase enables secure storage and efficient retrieval of data related to instructors, activities, and sessions.
    \item \textbf{Rendering Mode: Server-Side Rendering (SSR)} -- The project is configured to use Nuxt's server-side rendering mode, as specified in the configuration file. All routes are rendered on the server by default, providing improved SEO, performance, and dynamic content handling. This ensures that users receive fully rendered pages from the server, with client-side hydration for interactivity.
    \item \textbf{Main Framework: Nuxt} -- The application is built using Nuxt, a powerful Vue.js framework for building modern web applications. Nuxt provides features such as file-based routing, automatic code splitting, and an intuitive module system, streamlining development and improving maintainability.
    \item \textbf{Programming Language: TypeScript} -- TypeScript is used throughout the codebase to provide static typing, improved code quality, and better developer tooling. TypeScript helps catch errors early and enhances the maintainability of the project.
    \item \textbf{Component-Based Architecture} -- The project is organized using a modular, component-based structure, with reusable UI components for buttons, cards, carousels, and more. This promotes code reuse and simplifies updates and testing.
    \item \textbf{Version Control: GitHub} -- Source code is managed using Git and hosted on GitHub, enabling collaboration, version tracking, and integration with CI/CD pipelines.
\end{itemize}

\section{Project Structure}
% Overview of the project structure.

\subsection{Links/Pages Structure}

\begin{table}[H]
    \centering
    \caption*{\textbf{Website Navigation Structure}}
    \begin{tabular}{|p{2.5cm}|p{3cm}|p{8cm}|}
        \hline
        \rowcolor{bluepoli!20}
        \textbf{Page} & \textbf{URL} & \textbf{Description} \\
        \hline
        Home & / & Landing page with hero section, featured activities, teacher highlights, and call-to-action sections providing an overview of the yoga studio. \\
        \hline
        Highlights & /highlights & Displays featured yoga classes and special activities, including a prominently featured activity and a carousel of additional highlighted options. \\
        \hline
        Activities & /activities & Lists all available yoga classes with search functionality and filtering options. Users can browse and find detailed information about each activity. \\
        \hline
        Activity Detail & /activity/[id] & Shows detailed information about a specific yoga class, including description, difficulty level, duration, and teacher information. \\
        \hline
        Teachers & /teachers & Displays all yoga instructors with their specialties and a brief introduction. Users can click to view detailed profiles. \\
        \hline
        Teacher Profile & /teacher/[id] & Provides comprehensive information about a specific teacher, including their bio, expertise, contact information, and scheduled classes. \\
        \hline
        About Us & /about & Contains information about the yoga studio philosophy, mission, location, operating hours, and other facility details. \\
        \hline
        Contact Us & /contact & Features contact information, staff directory, location map, and a contact form for inquiries and class registrations. \\
        \hline
    \end{tabular}
    \caption{Website navigation structure showing main pages and their purposes}
\end{table}

\subsection{Available Server Endpoints}
% List and describe the server endpoints available in the project.

\section{Custom Types}
% Document the custom types defined for the project.

\section{Custom Components}
% Overview of custom components developed for the project.

\subsection{Buttons}
% Describe custom button components.

\subsection{Cards}
% Describe custom card components.

\subsection{Carousels}
% Describe custom carousel components.

\subsection{Containers}
% Describe custom container components.

\subsection{Slides}
% Describe custom slide components.

\subsection{Other Components}
% Describe any other custom components.

\section{Extra Modules}
% List and describe any extra modules developed.

\section{External Libraries}
% List and describe external libraries used in the project.

\chapter{Extras}
In this chapter, we will discuss how our website is compliant with the accessibility and
SEO guidelines. In order to evaluate such compliance, we have used the WAVE and
Lighthouse web evaluation tools.

\subsection{Accessibility Implementation}

Ensuring web accessibility was a core priority in the development of Lotus Haven's website to provide an inclusive experience for all users regardless of their abilities or disabilities. Our implementation follows the Web Content Accessibility Guidelines (WCAG) standards and has been validated using Lighthouse evaluation tools.

\subsubsection{Lighthouse Accessibility Evaluation}

According to our most recent Lighthouse report, Lotus Haven achieved an accessibility score of \textbf{96\%} which demonstrates our strong commitment to accessible design. This high score confirms that the website implements most of the recommended accessibility practices and ensures a smooth experience for users with various disabilities.

\subsubsection{Implemented Accessibility Features}

\paragraph{Semantic HTML Structure}
The website uses semantic HTML elements throughout to ensure that content is properly structured and navigable by screen readers and other assistive technologies:

\begin{itemize}
    \item \textbf{Proper heading hierarchy} with logical progression from H1 to lower-level headings
    \item \textbf{Landmark regions} such as navigation, main content, and complementary areas
    \item \textbf{Semantic elements} like \texttt{<section>}, \texttt{<article>}, and \texttt{<nav>} to organize content
\end{itemize}

\paragraph{Image Accessibility}
All images throughout the website follow accessibility best practices:

\begin{itemize}
    \item \textbf{Alternative text}: Informative images include descriptive alt text (e.g., "Lotus Haven Logo" for the site logo)
    \item \textbf{Decorative images}: Properly marked with empty alt attributes to prevent unnecessary screen reader announcements
    \item \textbf{SVG accessibility}: Icons include appropriate ARIA attributes when needed for context
\end{itemize}

\paragraph{Color Contrast and Visual Design}
The design ensures readability for users with visual impairments:

\begin{itemize}
    \item \textbf{WCAG AA compliant color contrast} between text and background
    \item \textbf{Visual indicators} beyond color alone for interactive elements
    \item \textbf{Consistent visual styling} for similar interactive elements
    \item \textbf{Customized scrollbars} with enhanced visibility for easier navigation
\end{itemize}

\paragraph{Keyboard Navigation}
The website is fully navigable without requiring a mouse:

\begin{itemize}
    \item \textbf{Logical tab order} following the visual flow of the page
    \item \textbf{Focus indicators} that are clearly visible when elements receive keyboard focus
    \item \textbf{Skip-to-content} functionality available for keyboard users
    \item \textbf{Interactive carousels} with keyboard controls (arrow navigation)
\end{itemize}

\paragraph{Responsive Design Considerations}
Accessibility extends to different viewport sizes and devices:

\begin{itemize}
    \item \textbf{Viewport meta tag} properly configured to support mobile devices and prevent zooming issues
    \item \textbf{Responsive layout} that adapts to different screen sizes
    \item \textbf{Zoom detection} that intelligently adapts the interface when users have zoomed in
    \item \textbf{Touch-friendly targets} of appropriate size for mobile users
\end{itemize}

\paragraph{ARIA Attributes}
ARIA (Accessible Rich Internet Applications) attributes are used judiciously to enhance accessibility:

\begin{itemize}
    \item \textbf{ARIA labels} for elements where the visible text is insufficient
    \item \textbf{ARIA roles} to clarify the purpose of custom widgets and components
    \item \textbf{Dynamic content updates} that are announced appropriately to screen readers
\end{itemize}

\paragraph{Forms and Interactive Elements}
All interactive elements are designed with accessibility in mind:

\begin{itemize}
    \item \textbf{Properly labeled form controls} with explicit associations between labels and inputs
    \item \textbf{Clear error messages} for form validation
    \item \textbf{Button states} (disabled, active, hover) that are clearly distinguishable
    \item \textbf{Focus management} for modal dialogs and other interactive components
\end{itemize}

\subsubsection{Implementation in Code}

Our accessibility features are not just superficial additions but are deeply integrated into the codebase:

\begin{itemize}
    \item All components include careful consideration of keyboard navigation, screen reader announcements, and visual focus states
    \item The \texttt{ClassesCarousel} and \texttt{TeachersCarousel} components implement enhanced keyboard accessibility and mouse wheel interaction for horizontal scrolling
    \item Viewport meta tags ensure proper display on mobile devices and prevent the 300ms tap delay
    \item Form components include built-in accessibility features like proper labeling and ARIA attributes
    \item The UI component library (\texttt{@nuxt/ui}) provides a foundation of accessible components that we've extended
\end{itemize}

\subsubsection{Remaining Accessibility Challenges}

While our Lighthouse score of 96\% is excellent, we acknowledge a few remaining areas for improvement:

\begin{itemize}
    \item Complex interactive elements like carousels could benefit from additional ARIA live regions for dynamic content updates
    \item Further enhancements to keyboard navigation in some complex widgets
    \item Additional testing with various screen readers to ensure consistent experiences
\end{itemize}

Our commitment to accessibility is ongoing, with regular testing and improvements planned as part of our development roadmap.

\subsection{SEO Compliance}

Search Engine Optimization (SEO) is crucial for ensuring that our website is discoverable by potential users through search engines. We've implemented several SEO strategies and best practices, validated through Lighthouse's SEO evaluation.

\subsubsection{Lighthouse SEO Evaluation}

The Lighthouse SEO analysis shows that our website follows modern SEO best practices, ensuring good visibility in search engine results. The implementation of proper meta tags, semantic HTML, and mobile-friendly design contributes to the website's search engine ranking potential.

\subsubsection{Implemented SEO Features}

\paragraph{Meta Tags and Document Structure}
We've implemented comprehensive meta tags across all pages to improve search engine understanding of our content:

\begin{itemize}
    \item \textbf{Title tags}: Each page has a unique, descriptive title that includes relevant keywords (e.g., "About Us | Our Studio \& Philosophy | Lotus Haven")
    \item \textbf{Meta descriptions}: Custom descriptions summarize each page's content and purpose for search engine results
    \item \textbf{Viewport meta tag}: Properly configured to ensure mobile-friendly display with "width=device-width, initial-scale=1"
    \item \textbf{Document} \texttt{<head>}: Well-structured with appropriate tags and prioritized loading
\end{itemize}

\paragraph{Semantic Content Structure}
Our content is organized with search engines in mind:

\begin{itemize}
    \item \textbf{Proper heading hierarchy}: Logical H1-H6 structure that helps search engines understand content importance
    \item \textbf{Descriptive link text}: Links use meaningful text rather than generic phrases like "click here"
    \item \textbf{Semantic HTML elements}: Content is structured with appropriate semantic tags that communicate meaning
    \item \textbf{Crawlable anchors}: All links are properly formatted to be discoverable by search engine crawlers
\end{itemize}

\paragraph{Social Media Integration}
We've enhanced social sharing capabilities through proper metadata:

\begin{itemize}
    \item \textbf{Open Graph Protocol}: Implementation of og tags (og:title, og:description, og:type, og:image) for optimized sharing on platforms like Facebook
    \item \textbf{Twitter Card markup}: Including twitter:card, twitter:title, and twitter:description tags for Twitter sharing optimization
    \item \textbf{Consistent branding}: Social metadata maintains consistent branding and messaging across platforms
\end{itemize}

\paragraph{Mobile Optimization}
Our mobile-first approach benefits SEO:

\begin{itemize}
    \item \textbf{Responsive design}: Content adapts seamlessly to various screen sizes
    \item \textbf{Touch-friendly elements}: Proper sizing and spacing of interactive elements
    \item \textbf{Fast loading on mobile}: Performance optimizations for mobile network conditions
\end{itemize}

\paragraph{Technical SEO Elements}
Several technical implementations enhance our SEO compliance:

\begin{itemize}
    \item \textbf{HTTPS implementation}: The entire site is served over secure HTTPS connections
    \item \textbf{Canonical URLs}: Properly defined to prevent duplicate content issues
    \item \textbf{Server-side rendering}: Nuxt's SSR capabilities ensure content is immediately available to search engine crawlers
    \item \textbf{robots.txt}: Properly configured to guide search engine crawlers
\end{itemize}

\subsubsection{Implementation in Code}

Our SEO features are implemented throughout the codebase:

\begin{itemize}
    \item \textbf{Dynamic meta tags}: Using Nuxt's \texttt{useHead} composable to set page-specific metadata based on content
    \item \textbf{Structured data}: Implementing JSON-LD for certain content types to enhance rich results in search engines
    \item \textbf{Image optimization}: Using Nuxt Image module for responsive image sizing and format optimization
    \item \textbf{SEO-friendly URLs}: Clean, descriptive URLs that follow best practices
\end{itemize}

Example from our \texttt{about.vue} page showing dynamic meta tag implementation:

\begin{verbatim}
useHead({
  title: "About Us | Our Studio & Philosophy | Lotus Haven",
  meta: [
    {
      name: "description",
      content:
        "Learn about Lotus Haven yoga studio, our philosophy, location, and mission. 
        Discover how we integrate ancient wisdom with modern practice to create 
        a nurturing wellness environment.",
    },
    { name: "viewport", content: "width=device-width, initial-scale=1" },
    { property: "og:title", content: "About Lotus Haven Yoga Studio" },
    {
      property: "og:description",
      content:
        "Learn about our yoga philosophy, studio location and hours at Lotus Haven.",
    },
    { property: "og:type", content: "website" },
  ],
});
\end{verbatim}

\subsubsection{SEO Monitoring and Future Improvements}

While our implementation of SEO best practices is comprehensive, we recognize that SEO is an ongoing process:

\begin{itemize}
    \item \textbf{Regular monitoring}: Tracking search performance through analytics tools
    \item \textbf{Keyword optimization}: Continuing to refine content based on relevant search terms
    \item \textbf{Performance optimization}: Further improving page load speeds to benefit search rankings
    \item \textbf{Structured data enhancements}: Expanding schema markup for more rich search results
\end{itemize}

Our approach to SEO is holistic, focusing not just on technical implementation but also on creating high-quality, relevant content that provides value to our users while following search engine guidelines.


\end{document}
